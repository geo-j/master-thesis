\section{Thesis Project Plan}
\label{sec:thesis}

The main goal of this thesis is to find an approximate solution to The Art Gallery Problem \cite{o1987art} using gradient descent. As such, we will devise an algorithm for optimising the positions of the guards inside the given Art Gallery polygon. 
The algorithm will be coded in C ++ with the help of the CGAL library(\url{https://www.cgal.org}).

\subsection{Motivation}
% see what Till says

\subsection{Progress So Far}
The first four weeks of the project have been spent on doing literature research.

Two weeks have been spent on creating the project coding skeleton. The skeleton was used to try out already given examples from the CGAL library (\url{https://www.cgal.org}). Visualisations for each optimisation step in the guard's positions have also been created using the Scikit Geometry library (\url{https://scikit-geometry.github.io/scikit-geometry/}).

The rest of the time has been spent on devising the gradient computation for one guard and implementing it.
The pseudo-code can be found in Algorithm \ref{alg:1g}.

\begin{algorithm}
    \begin{algorithmic}[1]
    \caption{Position Optimisation for One Guard}
    \label{alg:1g}
    \For{\textbf{each} guard $p(x, y)$}
        \State{\textit{cur\_guard\_position} $\gets (x, y)$}

        \While{\textit{prev\_guard\_position} $\neq$ \textit{cur\_guard\_position}} \Comment{continue as long as there are no more improvements in the guard's position}
            % \State{compute \textit{visibility\_region} of $p$}
            \State{compute $\bigtriangledown f$}
            \State{\textit{prev\_guard\_position} $\gets$ \textit{cur\_guard\_position}}
            \State{\textit{cur\_guard\_position} $\gets \textit{prev\_guard\_position} + l\bigtriangledown f$}

            \If{\textit{cur\_guard\_position} is outside the polygon}
                \State{place \textit{cur\_guard\_position} on polygon boundary}
            \EndIf
        \EndWhile
    \EndFor
    \end{algorithmic}
\end{algorithm}

The polygons used so far as input cases are the irrational guard polygon (Figure \ref{fig:p}), a star-shaped polygon (Subfigure \ref{fig:star}), a comb-shaped polygon (Subfigure \ref{fig:comb}), an arrow head (Subfigure \ref{fig:concave}) and an arbitrary polygon (Subfigure \ref{fig:random}). Each of them will be used for testing various aspects of the algorithm as follows:

\begin{itemize}
    \item the \textbf{Star polygon} will test whether guards can do a basic move from anywhere inside the polygon to the centre of it
    \item the \textbf{Arrowhead polygon} will test the movement of a guard to optimality on the boundary of the polygon
    \item the \textbf{Comb polygon} will test whether given 4 guards, the algorithm places each of them in one of the polygon spikes
    \item the \textbf{Arbitrary polygon} and the \textbf{Irrational Guards polygon} will combine the previously mentioned cases.
\end{itemize}

\begin{figure}
    \centering
    \begin{subfigure}{0.45\textwidth}
        \centering
        \includegraphics[width = \textwidth]{pentagram.png}
        \caption{Star test input polygon.}
        \label{fig:star}
    \end{subfigure}
    \begin{subfigure}{0.45\textwidth}
        \centering
        \includegraphics[width = \textwidth]{comb.png}
        \caption{Comb test input polygon.}
        \label{fig:comb}
    \end{subfigure}
    \begin{subfigure}{0.45\textwidth}
        \centering
        \includegraphics[width = \textwidth]{concave_triangle.png}
        \caption{Arrowhead test input polygon.}
        \label{fig:concave}
    \end{subfigure}
    \begin{subfigure}{0.45\textwidth}
        \centering
        \includegraphics[width = \textwidth]{random.png}
        \caption{Arbitrary test input polygon.}
        \label{fig:random}
    \end{subfigure}
    \caption{Input polygons used for testing the algorithm.}
\end{figure}

% So far:
% - literature research
% - visualisations
% - created the algorithm for one guard
% - implemented the gradient computation for one guard
% - get multiple polygons as test cases
% - have results for polygons that require rational coordinates

\subsection{Future Plans}
The next step in the development of this thesis is to extend and implement the gradient descent computation algorithm to multiple guards. That is, moving the guards interdependently.

Additionally, the algorithm will be tested with other polygons of different sizes to check its runtime performance. A comparison with other existing algorithms will be done at the end. The comparison will take place both in terms of runtime performance, as well as the optimised position of the guards. Because gradient descent is an approximation algorithm, the irrational guards' positions will be assessed using an error margin.

% Future plans:
% - algorithm for multiple guards
% - implement algorithm for multiple guards
% - test algorithm for multiple test cases 
% - compare with other algorithms (error margin)