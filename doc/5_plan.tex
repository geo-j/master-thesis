\section{Thesis Project Plan}
\label{sec:thesis}
The main goal of this thesis is to find an approximate solution to The Art Gallery Problem \cite{o1987art} using gradient descent.

\subsection{Progress So Far}
The first four weeks of the project have been spent on doing literature research (section \ref{sec:literature}). Then, three weeks have been spent on creating the project coding skeleton. The skeleton was used to try out already given examples from the CGAL library (\url{https://www.cgal.org}). Visualisations for each optimisation step in the guard's positions have also been created using the Scikit Geometry library (\url{https://scikit-geometry.github.io/scikit-geometry/}). The rest of the time has been spent on devising the gradient computation for one guard and implementing it, and writing the thesis report.
% So far:
% - literature research
% - visualisations
% - created the algorithm for one guard
% - implemented the gradient computation for one guard
% - get multiple polygons as test cases
% - have results for polygons that require rational coordinates

\subsection{Future Plans}
The next step in the development of this thesis is to extend and implement the gradient descent computation algorithm to multiple guards. That is, moving the guards interdependently. 

Then, a strategy for adding more guards when needed will have to be explored. When an optimum cannot be found with the current number of guards, we will add more, one by one. Multiple ways of doing so will be explored. Firstly, we can start from one guard. If no optimum can be found with it, we will add one more. We will continue adding guards until the whole polygon in question is fully visible. Conversely, we can also start from an arbitrary number of guards and remove or add more as needed. The latter option could be however more computationally expensive. This could be the case because when an optimum with a number of guards is found, we would still need to check whether it is the smallest possible number of guards.

Next to the strategy of new guard additions, we will need to devise a way to decide when an optimum cannot be reached. This will be a heuristic of observing whether we are stuck in either a local optimum, or are circling between multiple local optima.

Additionally, we will experiment of multiple ways of initial guard position. One such possibility would be to start with all guards on the same fixed starting position. Intuitively, this could deem suboptimal due to the fact that all guards would need to move away from each other anyway. Thus, a way to address this problem would be to start with the guards at arbitrary positions.

Lastly, the algorithm will be tested with other polygons of different sizes to check its runtime performance. A comparison with \cite{DBLP:journals/corr/abs-2007-06920} will be done at the end. The comparison will take place both in terms of runtime performance, as well as the optimised position of the guards. Because gradient descent is an approximation algorithm, the irrational guards' positions will be assessed using an error margin.

% Future plans:
% - algorithm for multiple guards
% - implement algorithm for multiple guards
% - test algorithm for multiple test cases 
% - compare with other algorithms (error margin) + Simon's
% - different ways of placing guards