\section{Discussion}
This thesis focussed on implementing and evaluating a gradient descent algorithm to find a solution to the Art Gallery Problem. This goal has been achieved for the few discussed polygons. 
Nonetheless, this method raised quite some issues. For this reason, we discussed more in-depth the development process and the performance of the algorithm. There were numerous edge-cases to be taken into account. For many of them, we created various hyperparameterised heuristics. Other heuristics (such as momentum) were created as an improvement to the shortcomings of gradient descent.
% Gradient descent is quite a straight-forward approach. 

The resulting program is sensitive to hyperparameter choices, polygon shapes and starting guard positions. For this reason, we only provided qualitative evaluations for the heuristics used to extend gradient descent. Namely, we discussed both specific added benefits of each of the heuristics, and a comparison based on the average number of iterations required to fully see a polygon. The way this was done is by running the program with all heuristics but the one whose practicality we are discussing. In this way, we assess the added advantage of each heuristic in its absence. The reason behind this experimental setup is that combinations of all heuristics are both verbose and unnecessary. Given that the program is still limited to a few examples, we cannot use extensive statistical tools to test its overall performance. With the given examples, we managed to assess that using the pull heuristic is likely to be statistically significant to need a lower number of iterations needed to fully see a polygon, despite its shape.
In this way, we provided a pragmatic overview to the usefulness of each of the heuristics and hyperparameters used. Namely, we analysed on what the type of polygons different heuristics and hyperparameter values provide concrete results.

Section \ref{sec:future} further discusses how the program could be improved in the future.



% - evaluation of what I did
%     - selection of heuristics that make sense (because their combination is too long)
% - "conclusion"

\subsection{Future Work}
\label{sec:future}
Currently, the program offers a state-of-the-art view about its practical possibilities.
In the future, it would be suitable to extend it with more robust functionalities.

One of the most crucial aspects of improvement for the program would be to solve the existing errors and bugs. As of now, the program crashes for specific input polygons (see Section \ref{sec:problems} for an extensive overview of its shortcomings). Another important element to consider would be to code it more efficiently, as the program does not scale. This is because some of the data structures and techniques used are naive (brute-forced). Implementing them in a more efficient way should improve the scalability of the program.

Additionally, some features and heuristics are not complete. For example, the greedy initialisation of the guards' position is deterministic. In order to quantitatively assess the performance of the algorithm, a truly randomised greedy initialisation would be required for statistical significance.

Lastly, it would be worth exploring how the algorithm would benefit from a different implementation. Namely, using another programming language like Python (or CGAL Python bindings) and other geometry libraries which are better documented and more reliable to use.
% - improve the efficiency of the code
%     - can add a flag for guards whose gradient has been computed instead of copying the vector
% - find a more efficient way of coding to tackle the edge-cases
% - implement an expiry date for the reflex area
% - implement algorithm for love polygon - currently crashes 
% - init gradient is not randomised -> should do that for finding actually working starting positions
% - give an impression on the code improvement


\newpage
\thispagestyle{empty}
\section*{Acknowledgements}
I would like to thank Till Miltzow and Frank Staals for supervising the thesis and always offering creative and constructive feedback. I would also like to thank Simon Hengeveld for the helpful advice and explanation of his implementation for solving the Art Gallery Problem.
Lastly, I would like to thank my partners Tim and Teun for always supporting me during both the rough and the happy times of this thesis.