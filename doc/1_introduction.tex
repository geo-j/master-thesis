\section{Introduction}
% Introduction and research questions: What is the problem? Illustrate with an example. What is/are your research questions/contributions? 
\subsection{The Art Gallery Problem}

The Art Gallery Problem \cite{o1987art} is a central problem in computational geometry. It can be introduced as follows: given a simple polygon $P$ with $n$ vertices, we are interested in finding the minimum number of guards that are able to see the whole polygon. A simple polygon is a polygon that has no holes. Thus, we can define the visibility of a guard $g$ in the polygon $P \subset \mathbb R^2$. The guard $g = (x, y) \in P$ sees another point $q \in P$ if the line segment $\overline{gq} \subseteq P$. The points that are visible from $g$ form the visibility polygon (region) $\mathit{Vis}(g)$. In the Art Gallery Problem, we are looking for a minimum size set of guards$~S$ that can see the whole polygon $P$.

Figure \ref{fig:art} displays an example of the Art Gallery Problem with polygon $P$ guarded by 2 guards ($|S| = 2$). The visibility region $\mathit{Vis}$ of each guard is marked with a different colour. For guard $g \in S$, its visibility region $\mathit{Vis}(g)$ is emphasised with the pink contour. The vertex $r$ blocks part of the view of $g$. The vertex $r$ is thus called ``reflex'', because its interior angle is larger than $180^\circ$. Reflex vertices are only found in concave polygons, so convex polygons can be guarded by only one guard.

\begin{figure}[h!]
    \centering
    \includegraphics{theory/visp.pdf}
    \caption{Example of an Art Gallery Problem instance with polygon $P$ guarded by 2 guards. The visibility area $\mathit{Vis}(g)$ is emphasised in pink.}
    \label{fig:art}
\end{figure}

The Art Gallery Problem is $\exists \mathbb R$-complete \cite{abrahamsen2021art}, which means it is even harder to solve than NP-complete problems \cite{schaefer2009complexity}. For this reason, approximation algorithms have been extensively used to address it (\cite{DBLP:journals/corr/BonnetM16b}, \cite{GHOSH2010718}, \cite{DBLP:journals/corr/abs-2007-06920}). Nonetheless, to the best of our knowledge there is no related work on approaching the Art Gallery Problem using gradient descent. As such, we approach the Art Gallery Problem from a new perspective using gradient descent.

\newpage
\subsection{Gradient Descent}

Gradient descent is an iterative optimisation algorithm for finding the minimum of a continuous differentiable function. The core idea of gradient descent is to repeatedly move in the opposite direction of the gradient of the function at the current point using a specific step size (learning rate). High learning rates result in approaching a local optimum faster, but risk overshooting it. Conversely, small learning rates are more precise, with the compromise of a longer computation and convergence time.
When there is no more change in the gradient, then an optimum has been reached. Gradient descent does not guarantee that the found optimum is global. For this reason, it can remain stuck in local optima.

Figure \ref{fig:gradient_descent} illustrates an example of applying gradient descent. The optimisation function takes the shape of a curve. Starting from an arbitrary point on the curve, the goal is to reach the minimum of the function (the bottom of the curve). This is done by computing the gradient (derivative) of the function at the current point. The gradient is displayed in red and tells that the largest increase in value for the function is going up the curve. Because we are interested in finding the minimum of the function, we move in the opposite direction of the gradient. So, we move down the curve with the given step size. Continuing from the new point on the curve, the process is repeated until the minimum is reached.
If we were interested in finding the maximum of the function, we would move in the same direction with the gradient.

\begin{figure}[h!]
    \centering
    \includegraphics{gradient.pdf}
    \caption{Illustrative example of applying gradient descent.}
    \label{fig:gradient_descent}
\end{figure}

\subsection{Thesis Goal}
This thesis is an optimisation algorithm engineering paper. The main goal is to create and implement an algorithm that uses gradient descent to approximate the solution to the Art Gallery Problem. The explored research question is whether the Art Gallery Problem can be efficiently solved using gradient descent. Additional to gradient descent, the algorithm deploys different heuristics to address various shortcomings and edge-cases. These are discussed based on polygon shapes.
% As such, we expect to be able to provide convergence guarantees for the algorithm. 
The algorithm is implemented in C++ using the CGAL library \cite{cgal}.

This paper is organised as follows: Section \ref{sec:literature} offers an overview of existing work. Section \ref{sec:theory} explains how gradient descent is used to solve the Art Gallery Problem and what heuristics we can make use of to improve the performance of our algorithm. Section \ref{sec:experiments} offers an overview of our algorithm's performance. Due to the time constraints of the thesis, extensive experiments have only been performed for one of the heuristics. Namely, Section \ref{evaluation} shows a more statistically significant preliminary evaluation for the experiments. For the rest, only an intuition about the advantages and use-cases for each of the heuristics is given. 
Lastly, Section \ref{sec:problems} discusses issues this project has encountered.
% As such, preliminary research and its implementation is presented as preparation for the second phase of the thesis project. Section \ref{sec:literature} offers an existing literature overview. Next, Section \ref{sec:theory} describes the relevant theory in detail.
% As a preview, Section \ref{sec:experiments} shows some introductory algorithm implementations and their performance.
% Lastly, Section \ref{sec:thesis} presents the development plan for the second phase of the Master's Thesis project.
