\section{Practice}
\label{sec:experiments}

This section will provide the implementation details of the theory part (Section \ref{sec:theory}). The algorithm is implemented in C ++ and makes use of the CGAL library (\url{https://www.cgal.org}). 
Then, given the implementation, experimental setups will be introduced. The experimental setup will include some basic test polygons to check the correct execution of the algorithm. We will also observe the behaviour of the algorithm and emphasise the importance of each of the heuristics used. Lastly, we will showcase the fragility and importance of hyperparameters.

Polygons used:
- 2 guards
- random
- love
- comb
- corridor

For every heuristic:
- show how each polygon behaves when turning it off

Try to see if the comb polygon scales

\subsection{Heuristics}
In this subsection we will observe the role played by each of the heuristics used. We will additionally notice how different heuristics are more relevant for different types of polygons. In order to do so, we will run the program with all of the heuristics but one, for each of the heuristics. By analysing the difference in movement for each of the guards', we will be able to assess the influence every heuristic has on each type of polygon.

We will use fixed hyperparameters for all the runs. This will allow us to focus on the differences between the heuristics. Later in this section we will also discuss the actual hyperparameter choice. As such, we will use a learning rate $\alpha = 0.4$, momentum past weight $\gamma = 0.8$, and pull attraction $\beta = 1$.

\subsubsection{Without Angle Behind Reflex Vertex}

- momentum
- reflex vertex pull
    - onto reflex vertex
    - pull Capping
- line search
- reflex area
- hidden gradient
- angle behind reflex vertex
- greedy initialisation