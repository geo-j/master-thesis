\documentclass{article}
\usepackage[utf8]{inputenc}
\usepackage{hyperref}


\usepackage{enumitem}



\begin{document}



\section*{Reflection by \textit{Georgiana Juglan}}



\begin{itemize}[leftmargin = 3cm, itemsep = 0cm]
    \item[\textbf{duration}:] 10 months
    \item[\textbf{year}:]  21/22
    \item[\textbf{topic}:] Solving the Art Gallery Problem Using Gradient Descent
    \item[\textbf{supervisor}:] Tillmann Miltzow
    \item[\textbf{contact possibilities}:] +40755101932, \href{mailto:g.juglan@tutanota.com}{g.juglan@tutanota.com} 
\end{itemize}



\paragraph{Project Summary.}
\textit{Try to summarize, both the content of the project, as well as the type of interaction, meetings, feedback, communication, expectations}

The project was an algorithm engineering paper. This meant that together with Till we devised an algorithm, and I implemented it in C++. 
We created a state-of-the-art algorithm that is able to solve the constrained fixed number of guards version of The Art Gallery Problem for a limited number of polygons.

We met weekly with all of the supervised students of Till to discuss my coding progress and receive feedback on it. Occasionally, we would exchange short messages with brief questions in between these meetings. After every meeting, I would prepare a to-do list for the respective week, as well as for the overview of the whole thesis. In this way, we could clear up the expectations that I had of Till for that respective week, and his of me.

I am overall happy with how easily and well the communication between me and Till went. He was always responsive to my questions, and I to his feedback. Sometimes we would hang out outside the regular meetings, which created a friendly environment to work in.




\paragraph{Strengths.}
\textit{What did you like best? What should be kept for the next time?}

I enjoyed Till's relaxed way of supervising, as well as his responsiveness. By being so, he created a friendly, comfortable and reliable studying environment for me. Namely, I felt like I could always count on receiving feedback and discuss with Till when needed. 
He always tried to help me himself, or point me to someone whom I could get help from. This made me feel like I could count on Till and not feel like I'm struggling alone through the thesis.

I was always very enthused by how creative and ecstatic Till was about the subject of my thesis. His interest in the problem was always a motivational boost, as well as contributed to the feeling of team work. Namely, I always felt that the subject mattered. Especially in struggling times during the thesis, this bigger picture about the importance of The Art Gallery Problem was a tremendous help to keep going.

I also enjoyed the social aspect that he created, by introducing his supervised students to each other, and allowing us to be in contact with each other. This helped with the lonely study environment that working on the thesis is sometimes. This also allowed for peer-reviewing, which I considered to be a very valuable part of the writing process.


\paragraph{Possible Improvements.}
\textit{What would you have wished for?
What would you do better next time?
What could I have done better?
What advice would you want to give past self one year ago?}

I am generally content with how the thesis supervision went. However, I would have some points of improvements in the setting goals areas. I felt like around halfway through the thesis I lost somewhat the big picture.
I started focussing on implementing more and more heuristics without thoroughly testing the already existing one. In this way, I followed a broader approach to solving the problem. 

Next time I would create a more thorough procedure to test my algorithm. In particular, I would try to make it more modular so that I can create unit tests for specific gradient descent computations. So, I would test the base algorithm more robustly before implementing new functionalities on top of it. I would also refactor the code more frequently. In the second half of the thesis I got stressed about my progress and started to be less strict about the structure of my code, while trying to implement as many heuristics as possible. I believe that by keeping to a stricter coding and testing procedure, I would go more in-depth with only a few of the heuristics implemented, but the end result would be more robust and reliable.

I would also phrase and voice my concerns more clearly regarding the evolution of my thesis. For example, I would discuss whether a broader or a deeper approach would be more favourable for the specific stage of my thesis. I would also weigh better all the ideas that Till had. I was excited to implement and assess how the algorithm would behave with different new heuristics. However, next time I would be more conservative with the implementation of new features before thoroughly assessing the older ones. However, this is sometimes harder to become aware of when in the process of the thesis rather than at the end of it.

In hindsight, I realise that Till emphasised to extensively check that the gradient descent computations are done correctly. Nonetheless, I was sometimes narrowly focussed on specific edge-cases that he was pointing out, trying to solve those and not looking beyond them. As previously mentioned, next time I would contain myself to doing one heuristic at a time well, rather than multiple ones incompletely. I believe that it were useful if Till also drew attention to the deeper vs. broader approach, rather than frequently coming up with new ideas. However, this feels more of a development ``boundary'' balancing from both sides rather than only from one side.

Moreover, I would have liked if Till held more tightly to his deadlines. This was not necessarily an issue during the thesis, as I was mostly flexible and alright with reminding him about it. It would have been nice nonetheless to receive some last feedback before officially handing in the thesis. In the end, this was still not a big issue and I could finish my thesis on time and within the requirements.



\paragraph{Autonomy and Guidance.}
\textit{The ideal supervisor gives advice to the student when needed but otherwise as much autonomy as possible.
Did you feel that you had enough autonomy? Did you feel that you were given enough advice?}

Yes, I feel like generally the balance between advice and autonomy was proper. The main thing worth noting, that I have already mentioned, is that it would have been useful to find a better balance between my coding responsibilities and boundaries as a developer and Till's heuristic ideas. Otherwise, I feel like our interactions and meetings were always useful and insightful. I had enough autonomy to schedule my work as I needed, while still keeping the meetings as a weekly guidance point.


\paragraph{Challenge Level.}
\textit{Did you feel that the project had challenges at the right level for you?
If no, where the tasks rather too difficult or too easy for you?}

I feel like overall the project was at the right level for me. I enjoyed taking on the challenge of properly organising a new coding project from scratch and learning to use more committedly different software development tools (e.g.: Git).

The main challenge was comprised by the CGAL library. Because of the lack of online and social resources, it sometimes felt mind-breaking, desperate and like a waste of time to try to debug the cryptic errors that I could not figure out myself.

Another challenge that I faced was that of properly writing Mathematical statements. The final feedback about the critical approach and phrasing of the theory part came partially as a surprise, especially with the theory section having been read and given feedback on by Till. In this case, I would not know how to more concretely improve my theoretical Mathematical writing.



\paragraph{Social Learning.}
\textit{Did you feel that there was some sort of community who gave you social support during your thesis process?}

I was happy to be in the community of Till and his other supervised students. I appreciated the weekly meetings, general talk about the thesis and our situation, and the few helpful meetings with older students of Till. However, I felt like bigger/more consistent social support was missed.


\paragraph{Communication.}
\textit{How did you like the communication in general?}

I am happy with how the communication between both me and Till and his other students went. I am grateful that he was always so fast to respond, and did his best to make himself available when needed.
I am also happy with Till's friendliness and openness to discuss both thesis-related and -unrelated topics.


\end{document}