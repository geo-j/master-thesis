\section{Literature Summaries}
\label{sec:literature}
% Scientific back ground : What is the existing technology and literature that you will be studying/using in my research 

This section offers an overview of previous works addressing the Art Gallery Problem \cite{o1987art}. 
Working with the visibility region is a basic tool in computational geometry, specifically in the context of the Art Gallery Problem \cite{o1987art}, which is an $\exists \mathbb R$-complete problem \cite{abrahamsen2021art}. The problem class $\exists \mathbb R$ consists of instances that can be reduced in polynomial time to a decision problem of whether a system of polynomial equations with integer coefficients and any number of real variables has a solution. Since NP $\subseteq \exists \mathbb R$, the $\exists \mathbb R$ class is an even harder to solve complexity class, since $\exists \mathbb R$-complete problems are not know to have solutions of polynomial space. 
% For this reason, it is crucial that the computation time of the visibility region is efficiently implemented. 

Building on the aforementioned concepts, this section will further inspect how visibility in polygons can be efficiently computed \cite{DBLP:journals/corr/BungiuHHHK14}, how it is not always possible to place guards at rational coordinates \cite{abrahamsen2021art}, in addition to two improved polygon guarding algorithms \cite{maleki2022implementation}, \cite{DBLP:journals/corr/abs-2007-06920}.
% Some general notions that will appear in the papers are introduced below.

% Visibility in simple planar polygons can be defined as: given a polygon $\mathcal P \subset \mathbb R^2$, a point (guard) $p \in \mathcal P$ \textit{sees} a point $q \in \mathcal P$ if the line segment $\overline{pq} \subseteq \mathcal P$. Thus, the points that are visible from $p$ form the visibility polygon (region) $Vis(p)$.

% A guard set $S$ is defined as a set of points in a given polygon $\mathcal P$ such that every point in $\mathcal P$ is seen by some point in $S$. $\forall x, y \in \mathcal P$, $x$ sees $y$ if the line segment $\overline{xy} \in \mathcal P$. 

\subsection{Efficient Computation of Visibility Polygons}
Bungiu, Hemmer, Hershberger, Huang and Kr\"oller  \cite{DBLP:journals/corr/BungiuHHHK14} introduce the implementations and their experimental evaluations for two existing algorithms (\cite{joe1987corrections}, \cite{asano1985efficient}) and a newly developed one for computing visibility in polygons. These implementations are available in the CGAL library (\url{https://www.cgal.org/}), starting with version 4.5.


Therefore, Bungiu et al. present three algorithms and their practical performance.

\subsubsection{Algorithm of Joe and Simpson}
The algorithm of Joe and Simpson \cite{joe1987corrections} runs in $O(n)$ time and space. 

Let $v_i$, for $i = {1, 2, ..., n}$, be the boundary vertices of the polygon $P$. Let $p$ be a guard in $P$, and let $s$ be a stack datastructure. The stack $s$ will be used to keep track of the vertices determining $Vis(p)$. 

The algorithm begins by scanning the boundaries of $P$. The scanning is done through shooting rays $\vec{pv_i}$, for $i = {1, 2, ..., n}$ in this order. The endpoints $v_i$ and $v_{i + 1}$ of each ray form a boundary edge $\overline{v_iv_{i + 1}}$. In this way, the processing of $v_i$ and $v_{i + 1}$ is done by checking whether the points are in $Vis(p)$. This means that the position of every $v_{i + 1}$ with respect to the ray $pv_i$ is checked. If $v_{i + 1}$ is found in front of the ray $\vec{pv_i}$ (if $v_{i + 1}$ is seen from $p$), then both $v_i$ and $v_{i + 1}$ are added to the processing stack $s$. For every newly pushed vertex on $s$, the algorithm checks whether the segment $\overline{v_iv_{i + 1}}$ obscures any of the previously added vertices. If that is the case, then the endpoints of the obscured line segment are declared obsolete and deleted. The polygon comprised of the vertices from $s$ forms at the end the visibility polygon $Vis(p)$.

% TODO: this might need better explanation? not sure if edge cases are needed
Figure \ref{fig:joe} displays an example run of the Algorithm of Joe and Simpson \cite{joe1987corrections} for polygon $P$ and guard $p$. First, the ray from $p$ is shot through vertex $v_1$, and $v_1$ is added to $s$. Then, the ray from $p$ is shot through $v_2$. Since ray $\vec{pv_2}$ takes a right turn from $\vec{pv_1}$, this means that $v_2$ is still in the visibility region of $p$. For this reason, $v_2$ is also added to $s$. The ray passing through $v_2$ also intersects the boundary of $P$ in a point $v_2'$. To account for the fact that $p$ can see ``behind'' $v_2$ and is still inside $P$, the boundary vertex $v_2'$ is hence added to $s$. Next, the ray $\vec{pv_3}$ takes a left turn from $v_2$, which means that $v_3$ is not seen from $p$. Similarly, $v_4$ and $v_5$ are added to $s$. However, because ray $\vec{pv_6}$ takes a left turn from $\vec{pv_5}$, segment $\overline{v_5v_6}$ obscures $\overline{v_4v_5}$. So, $v_4$ and $v_5$ are removed from $s$. Ray $\vec{pv_6}$ then intersects the boundary of $P$ in $v_6'$. At the end, $Vis(p) = \{v_1, v_2, v_2', v_6, v_6', v_7, v_8, v_9, v_{10}\}$, as shown on the boundary of the green area. 

\begin{figure}[h!]
	\centering
	\includegraphics[width=0.6\textwidth]{literature/joe_simpson_eg.png}
	\caption{An example run of the algorithm of Joe and Simpson \cite{joe1987corrections} for polygon $P$ guarded by $p$ and boundary vertices $v_i$, for $i = \{1, 2, ..., n\}$.}
	\label{fig:joe}
\end{figure}

% boundaries of t	he simple polygon $P$, and adds its boundary points $v_i, \forall i = \overline{1, n}$, with $n$ the number of vertices in $P$, to a stack $s$. For each processed edge $\overline{v_iv_{i + 1}}$, its endpoints $v_i$ and $v_{i + 1}$ are checked whether they are in the visibility region of the viewpoint $p$. If they are, $v_i$ and $v_{i + 1}$ are added to~$s$. Otherwise, they are skipped. At every moment, the algorithm checks whether $\overline{v_iv_{i + 1}}$ obscures a previously added line segment. If that is the case, then the endpoints of the obscured line segment are declared obsolete and deleted. 

% The implementation of the algorithm handles the previously discussed cases for an arrangement $P$, while also accounting for the case in which the polygon winds more than 360$^\circ$ using a winding counter.
% - **Algorithm of Joe and Simpson** $O(n)$ time and space
	% - performs a sequential scan of the boundary of $P$ and uses a stack $s$ of boundary points $s_0, s_1, ..., s_ as summarised in the following subsections.der to deal with cases in which the polygon winds more than 360*, a winding counter is used during this edge processing
% he points that are visible from $q$ form the visibility region $\mathcal V(q)$ (polygon)
\newpage
\subsubsection{Algorithm of Asano}
The algorithm of Asano \cite{asano1985efficient} runs in $O(n \log n)$ time and $O(n)$ space. It uses a plane sweep approach with event line $L$. 

Let $P$ be a polygon determined by vertices $\{a_1, a_2, a_3, b_1, b_2, b_3, b_4\}$. Suppose we want to guard it by point $p$. The algorithm of Asano \cite{asano1985efficient} begins by efficiently sorting all the vertices of $P$ based on their polar angles with respect to the guard $p$. Figure \ref{fig:asano_1} displays an example run of the algorithm. The points will be treated in the order of $a_2, a_1, a_3, b_4, b_3, b_2, b_1$ with respect to $p$ and their angular comparisons $$\measuredangle a_2Op < \measuredangle a_1Op < \measuredangle a_3Op < \measuredangle b_4Op < \measuredangle b_3Op < \measuredangle b_2Op < \measuredangle b_1Op,$$ where $O = (0, 0)$.


Then, the event line $L$ starts sweeping around $p$ as shown in Subfigures \ref{fig:asano1} - \ref{fig:asano7}. Every line segment that $L$ intersects is stored in a balanced binary tree $T$ in the order of their intersection angle. As $T$ is updated, a new vertex of $Vis(p)$ is stored each time the segment closest to $p$ in $T$ changes. It is important to mention that the intersection between $L$ and the line segments is not explicit, but is instead determined by comparisons between the endpoints' coordinates. For instance, in Figure \ref{fig:asano_1}, the endpoint $b_2$ of line segment $\overline{b_2a_2}$ is the first one $L$ intersects. Point $b_2$ is thus added to $T$. Then, $L$ continues sweeping and adds $b_1, a_2$ and $a_1$ to $T$. Although $\overline{b_2a_2}$ and $\overline{b_1a_1}$ represent line segments $s_2$ and $s_1$, respectively, the intersection of $L$ with them is not explicitly computed, but is determined based solely on the positions of their endpoints: $s_1$ is farther away from $p$ because $q, a_2$ and $b_2$ are on the same side of $s_2$.

\begin{figure}[h!]
	\centering
	\begin{subfigure}{0.45\linewidth}
		\includegraphics*[width = \linewidth]{literature/asano7.png}
		\caption{$a_2$ is added to the empty $Vis(p)$ such \\ that $Vis(p) = \{a_2\}$.}
		\label{fig:asano1}
	\end{subfigure}
	\begin{subfigure}{0.45\linewidth}
		\includegraphics*[width = \linewidth]{literature/asano1.png}
		\caption{$a_1$ is added to $Vis(p)$ such that \\ $Vis(p) = \{a_1, a_2\}$, and line segment $\overline{b_1a_1}$ is added to the empty binary tree $T = \{\overline{b_1a_1}\}$.}
	\end{subfigure}
	\begin{subfigure}{0.45\linewidth}
		\includegraphics*[width = \linewidth]{literature/asano2.png}
		\caption{$a_3$ is added to $Vis(p)$ such \\ that $Vis(p)~=~\{a_1, a_2, a_3\}$,  and line \\ segment $\overline{a_1a_3}$ is added to $T$ such that \\ $T = \{\overline{b_1a_1}, \overline{a_1a_3}\}$.}
	\end{subfigure}
	\begin{subfigure}{0.45\linewidth}
		\includegraphics*[width = \linewidth]{literature/asano3.png}
		\caption{$b_4$ is added to $Vis(p)$ such \\ that $Vis(p)=\{a_1, a_2, a_3, b_4\}$, and line \\ segment $\overline{a_3b_4}$ is added to $T$ such that \\ $T = \{\overline{b_1a_1}, \overline{a_1a_3}\, \overline{a_3b_4}\}$.}
	\end{subfigure}
	\begin{subfigure}{0.45\linewidth}
		\includegraphics*[width = \linewidth]{literature/asano4.png}
		\caption{$b_3$ is added to $Vis(p)$ \\ such that $Vis(p)=\{a_1, a_2, a_3, b_4, b_3\}$, \\ and line segment $\overline{b_4b_3}$ is added to $T$ \\ such that $T = \{\overline{b_1a_1}, \overline{a_1a_3}\, \overline{a_3b_4}, \overline{b_4b_3}\}$.}
	\end{subfigure}
	\begin{subfigure}{0.45\linewidth}
		\includegraphics*[width = \linewidth]{literature/asano5.png}
		\caption{$b_2$ is added to $Vis(p)$ such that \\ $Vis(p) = \{a_1, a_2, a_3, b_3, b_4, b_2\}$, and line segment $\overline{b_2a_2}$ is added to $T$ such that \\ $T = \{\overline{b_1a_1}, \overline{a_1a_3}\, \overline{a_3b_4}, \overline{b_4b_3}, \overline{b_2a_2}\}$.}
	\end{subfigure}
	\caption{Example run of the Algorithm of Asano \cite{asano1985efficient} on polygon $P$ and guard $p$. The vertices of $P$ are added to the binary tree $T$ in the order of their angle between $p$ and the origin $O = (0, 0)$. The result of the algorithm is visibility region $Vis(p) = \{a_1, a_2, a_3, b_3, b_4, b_2\}$.}
	\label{fig:asano_1}
\end{figure}
\begin{figure}[h!]
	\ContinuedFloat
	\centering

	\begin{subfigure}{\linewidth}
		\centering
		\includegraphics*[width = 0.6\linewidth]{literature/asano6.png}
		\caption{$b_1$ is not added to $Vis(p)$ because it is obstructed by the line segment $\overline{b_2a_2}$ which is already in $T$. For the same reason, line segments $\overline{b_3b_1}$ and $\overline{b_1a_1}$ are also not added in $T$.}
		\label{fig:asano7}
	\end{subfigure}
	\caption{Example run of the Algorithm of Asano \cite{asano1985efficient} on polygon $P$ and guard $p$. The vertices of $P$ are added to the binary tree $T$ in the order of their angle between $p$ and the origin $O = (0, 0)$. The result of the algorithm is visibility region $Vis(p) = \{a_1, a_2, a_3, b_3, b_4, b_2\}$.}
	\label{fig:asano_2}
\end{figure}



% \begin{figure}[h!]
% 	\centering
% 	\includegraphics[width = 0.5\textwidth]{compare_segments.png}
% 	\caption{The Algorithm of Asano \cite{asano1985efficient} Visual Example \cite{DBLP:journals/corr/BungiuHHHK14}.}
% 	\label{fig:asano}
% \end{figure}
	% - as the sweep proceeds, $T$ is updated and a neq vertex of $V(q)$ is generated each time the smallest element (segment closest to $q$) in $T$ changes
	% - important to have efficient comparison ops (e.g.: *add pic*)
\newpage 
\subsubsection{New Algorithm: Triangular Expansion}
The algorithm introduced by Bungiu et al. \cite{DBLP:journals/corr/BungiuHHHK14} is named Triangular Expansion and runs in $\Omega(n^2)$ time and $O(n)$ space. It begins by triangulating $P$ in $O(n \log n)$ time if $P$ has holes, and $O(n)$ otherwise. The implementation runtime is constrained by CGAL, which makes use of the Delaunay triangulation algorithm \cite{delaunay1934sphere} with $O(n^2)$ time for the worst case, but with better performance in practice. 

Taken from \cite{DBLP:journals/corr/BungiuHHHK14} and annotated to suit the explanations in these summaries, Figure \ref{fig:triangular} depicts an example run of the algorithm on a polygon with holes $P$. Starting from the viewpoint $p$, the triangle containing $p$ is located by performing a simple walk. Trivially, $p$ sees the entire triangle it is contained in. The algorithm continues by recursively expanding the view of $p$ from one triangle into the next, until there are no more triangles to expand into. The view of $p$ becomes restricted by the reflex vertices $l$ and $r$ of the third triangle entered by the recursive step. Since $l$ and $r$ are reflex vertices, the view past them is further restricted until the boundaries $l'$ and $r'$ of $P$, respectively,  are reached. Line segments $\overline{ll'}$ and $\overline{rr'}$ are added to $Vis(p)$ in their angular order around $p$. At the end, $Vis(p)$ will contain the segments delimiting the visibility polygon of $p$.

\begin{figure}[h!]
	\centering
	\includegraphics[width = 0.5\textwidth]{literature/triangular_expansion.png}
	\caption{The Triangular Expansion Algorithm Example - recursion entering triangle $\Delta$ through edge~$e$~\cite{DBLP:journals/corr/BungiuHHHK14}.}
	\label{fig:triangular}
\end{figure}
% - **triangular expansion** - $O(n^2)$
	% - preprocessing: triangulation ($O(n)$ for simple polygons, $O(n\log n$) for polygons with holes; Delaunay ($O(n^2)$) used)
	% - given $q$, locate the triangle containing $q$ by a simple walk ($q$ sees the entire triangle)
	% - recursive procedure that expands the view of $q$ through that edge into the next triangle. Initially, the view is restricted by the 2 endpoints of the edge, and then further as recursion continues: *add pic* for triangle $\Delta$, the view of $q$ is restricted by the 2 reflex vertices $l$ and $r$ with $a \leq r < l \leq b$ w.r.t. angular order around $q$. $v$ is a new vertex and its position w.r.t. $l$ and $r$ is computed with 2 orientation tests *add pic*: $e_l$ is a boundary edge and we can report edge $\overline{ll'}$ and $\overline{l'v}$ as part of the visibility region of $q$; $e_r$ is not a boundary edge => the recursion continues with $v$ being the vertex that now restricts the left side of the view
	% - the recursion may split into 2 calls if $e_l$ and $e_r$ are both not part of the boundary. As there are $n$ vertices, this can happen $O(n)$ times => worst-case $O(n^2)$; however a true split into two visibility cones that may reach the same triangle independently can only happen at a hole of $P$, thus at worst the runtime is $O(nh)$, where $h$ = number of holes (linear time of simple polygons) (e.g.: worst-case *add pic*)
	% - triangulation has linear size, at most $O(n)$ recursive calls on the stack => $O(n)$ space

\subsubsection{Experiments}
Bungiu et al. do not report on benchmarks with query points on edges in the interior polygon. This is because they claim that their implementations perform similarly to other already implemented algorithms. Instead, they use two real-world scenarios (a simple polygon of Norway with 20981 vertices, and a cathedral polygon with 1209 vertices) and a worst-case polygon for the Triangular Expansion algorithm.

In terms of results on the real-world polygons, the Triangular Expansion algorithm has a 2-factor improved performance when compared to Asano's algorithm \cite{asano1985efficient}, and performs ``one order of magnitude'' faster than Joe and Simpson's algorithm \cite{joe1987corrections}. For the worst-case scenario, Asano's algorithm \cite{asano1985efficient} outperforms the Triangular Expansion algorithm with increasing input complexity.

Thus, despite the Triangular Expansion algorithm being outperformed in the worst-case scenario, Bungiu et al. add efficient implementations for  3 different  polygon visibility algorithms in the CGAL library. The choice of algorithms when using the library can be adapted based on the input polygons. 
% - experiments - no reports on similar benchmarks with query points on edges and in the interior polygon; for the input graphs used, the triangular expansion is 2-factor faster than Asano, and one order of magnitude faster than Joe and Simpson; with increasing input complexity, Asano does become faster
\input{2b_summaries}
\newpage
\subsection[Implementation of Guarding Algorithms]{Implementation of Polygon Guarding Algorithms for Art Gallery Problems}
Maleki and Mohades \cite{maleki2022implementation} implement efficiently two existing approximation algorithms (\cite{GHOSH2010718}, \cite{bhattacharya2016approximability}) for computing visibility in simple polygons. Namely, they introduce practical implementations for visibility algorithms that already offer theoretical guarantees. Additionally, they develop their own visibility algorithm that aims to offer performance guarantees for polygon visibility computation. They lastly  evaluate experimentally their implementations for the three algorithms in the context of the Art Gallery Problem \cite{o1987art}.

To begin with, we will introduce some terminology used throughout this summary. The algorithms distinguish in their implementation between vertex guards and point guards in  a polygon $P$. Vertex guards can be placed only on the vertices of $P$. Point guards can be placed without restriction inside~$P$. Lastly, the algorithms are tested on weak visibility polygons. A polygon $P$ has weak visibility if all boundary vertices of $P$ can see all the points in $P$.

% Given that computing a minimum number of guards for guarding a polygon is NP-hard \cite{1057165}, Maleki and Mohades inspect how the Art Gallery Problem \cite{o1987art} can be tackled using three approximation algorithms.

\subsubsection{Algorithm of Ghosh}
The algorithm of Ghosh \cite{GHOSH2010718} runs in $O(n^4)$ time and yields an $O(\log n)$-approximation algorithm for computing the minimum vertex guard for simple polygons. 

One of the most important concepts the algorithm works with is that of a convex component. Given a polygon $P$, we can form subsets of the vertices in $P$ such that all the subsets form convex subpolygons of $P$. We call these subsets of vertices the convex components of $P$.

The algorithm begins by computing the set of all the convex components $C$ of a given polygon $P$. Then, each vertex $j \in P$ creates a set $F_j$ with the convex components from $C$ that it is able to fully see. Let $F := \{F_j \mid \forall j \in P, F_j \subseteq C \text{ seen by } j\}$. Then, the algorithm checks for overlaps between the sets of $F$. For every vertex $i$ and its corresponding convex components set $F_i$, the algorithm checks whether any of the other sets $F \setminus F_i$ sets are included in $F_i$. That is, if $F_j \subseteq F_i, F_j \in F, i \neq j$. If that is the case, then  vertex $i$ sees at least as little as $j$. The vertex $j$ is thus not needed as a guard, so $F_j$ is removed from $F$. Vertex $i$ is added in the final guard set $S$. The algorithm repeats until the set $F$ becomes empty. When that happens, it means that the algorithm found a set $S$ of guards who see all the convex components of $P$ without overlap.

% The redundant sets $F_j, \forall j$ are afterwards eliminated as follows: for all fixed $j$, the algorithm searches for a vertex $i \neq j, i \in P$ such that $F_i$ is also visible from $j$ ($|F_i| \leq |F_j|$); $F_i$ is eliminated, $j$ is added to $S$ and the process continues until $C$ is empty. At the end, $S$ contains the approximated guard set for $P$.

An example run of the algorithm is depicted in Figure \ref{fig:ghosh}. Let $P$ be the polygon in question, and its boundary vertices $\mathcal V = \{1, 2, 3, 4, 5, 6\}$. Polygon $P$ is divided into two convex components $C_1$ and~$C_2$, such that $C = \{C_1, C_2\}$. The sets $F_j, \forall j \in P$ are also displayed. Since $F_1 \subseteq F_2 \subseteq F_3 \subseteq F_4 \subseteq F_6$ and~$F_5 \subseteq F_3 \subseteq F_4 \subseteq F_6$, then $F_1, F_2, F_3, F_4$, and $F_5$ are removed. The remaining set $F_6$ yields the final guard set $S = \{6\}$, which can see both convex regions of $P$, and hence the whole polygon.

\begin{figure}[h!]
    \centering
    \includegraphics{literature/ghosh_ex.pdf}
    \caption{Example run of the Algorithm of Ghosh \cite{GHOSH2010718} with polygon $P$ divided into two convex components $C_1$ and $C_2$. The resulting guard set is $S = \{6\}$.}
    \label{fig:ghosh}
\end{figure}

\newpage
\subsubsection{Algorithm of Bhattacharya, Ghosh and Roy}
The algorithm of Bhattacharya et al. \cite{bhattacharya2016approximability} runs in $O(n^2)$ time and yields a 6-approximation for computing the minimum vertex guard for weak visibility polygons without holes. 

It begins by choosing two neighbours $u$ and $v$ as parents for every vertex in $P$. It then computes the Shortest Path from each pair of parents $(u, v)$ to every other vertex in $P$. The Shortest Path is a path between two vertices such that the distance between them is minimal. If all distances between two adjacent vertices are the same, then the length of the path corresponds to the number of edges between the vertices. The Shortest Path from a pair of vertices $(u, v)$ to any other vertex $w$ is the length of the minimum path between $u$ and $w$, and $v$ and $w$.

Then, all vertices that can be reached from $u$ and $v$ are unmarked. In increasing angular order from $\overline{uv}$, every vertex $w \in P$ is checked for visibility from $u, \text{ and } v$. If all vertices $w$ are visible, then $u, \text{ and }v$ are added to $S$. Otherwise, $w$ is added to $S$ and the procedure continues with $w$ as the starting node. All vertices that become seen from $S$ are marked. At the end, the algorithm checks whether the vertices in $S$ have overlapping visibility regions, and duplicates are removed.

An example run of the algorithm is depicted in Figure \ref{fig:bhaca}. Let $P$ be the polygon in question, and its boundary vertices $\mathcal V = \{a, b, c, d, e, f\}$. The algorithm starts with vertices $a$ and $c$ as parents of $b$. Clockwise, vertex $d$ is visible from $\overline{ac}$, but $e$ is not. For this reason, $e$ is added to the guard set $S$, and vertices $d$ and $f$ are chosen as the new parents. In increasing angular order, all vertices $a, b, c, e$ are visible from $\overline{ef}$, so $d$ and $f$ are added to $S$. Since the visibility regions of $d$ and $f$ coincide~($\mathit{Vis}(d) = \mathit{Vis}(f)$), and the visibility region of $e$ is included in that of $d$ and $f$ ($\mathit{Vis}(e) \subseteq \mathit{Vis}(d) \subseteq \mathit{Vis}(f)$), $e$ and $d$ can be removed from $S$. As such, the final guard set becomes $S = \{f\}$.

\begin{figure}[h!]
    \centering
    \includegraphics{literature/bacha_ex.pdf}
    \caption{Example run of the Algorithm of Bhattacharya et al. \cite{bhattacharya2016approximability} for polygon $P$. The final guard set is $S = \{f\}$.}
    \label{fig:bhaca}
\end{figure}

\newpage
\subsubsection{New Algorithm}
The algorithm introduced by Maleki and Mohades is focussed on polygons with large number of vertices~$n$ and different amounts of reflex vertices $r$. If the number of reflex vertices is significantly lower than the total number of vertices ($r \leq \log \log n$), guards are placed at all reflex vertices. Otherwise, they are placed according to the algorithm of Ghosh \cite{GHOSH2010718}.

\subsubsection{Experiments}
Algorithms of Ghosh \cite{GHOSH2010718} and Bhattacharya et al. \cite{bhattacharya2016approximability} are tested on weak visibility polygons, and the newly introduced algorithm is tested on simple polygons. 

Let Procedure 1 be a procedure for generating weak visibility polygons that is illustrated in Figure~\ref{fig:weak}: given two points $p = (k, 0), q = (-k, 0)$,  $n$ random points $\{x_1, ..., x_n\}$ sorted on the distance from $p$ on $\overline{pq}$, $n$ sorted random angles $\{\alpha_1, ..., \alpha_n\} \in  (0, \pi)$, and $n$ vertices $\{y_1, ..., y_n\}$ are created by shooting $n$ rays at the corresponding angle $\alpha_i, \forall x_i$. Then, $n$ reflex vertices $\{z_1, ..., z_n\}$ are added in the quadrilateral formed by vertices $x_iy_iy_{i + 1}x_{i + 1}$. The figure is accredited to \cite{maleki2022implementation}.

\begin{figure}[h!]
    \centering
    \includegraphics[width=0.6\textwidth]{literature/weak.png}
    \caption{Generated weak visibility polygon for $n = 3$ \cite{maleki2022implementation} through Procedure 1.}
    \label{fig:weak}
\end{figure}

The algorithms of Ghosh \cite{GHOSH2010718} and Bhattacharya et al. \cite{bhattacharya2016approximability} were tested using polygons generated by Procedure 1 with $n \in \{10, 11, ..., 15\}$ vertices and $r \in \{2, 3\}$ reflex vertices. The results suggest that for low values of $n$ and $r$, the algorithm of Ghosh \cite{GHOSH2010718} performs better when using the number of guards as evaluation criteria. 
% Its constant time approximation in the small case contrasts with the constant approximation of the algorithm of Bhattacharya \cite{bhattacharya2016approximability}.
Similarly, the algorithm of Ghosh \cite{GHOSH2010718} performed better both when tested on a weak visibility polygon with low $n = \overline{10, 15}$ and $\frac n 2 \leq r$, and when tested with large $n \in \{100, 400\}$.

Secondly, let Procedure 2 be a procedure for generating simple polygons with custom number of reflex vertices $r$ is devised in Figure \ref{fig:arbitrary}. The figure was taken from \cite{maleki2022implementation} and annotated to suit the explanations in these summaries. Starting from a simple convex polygon $P$ with $n$ vertices~$u, v, x1, x2, ..., x7$, polygon $P$ is triangulated such that every triangle has a joint edge with its boundaries;~$r$ reflex vertices $r1, r2, r3$ are randomly added inside $P$, and the boundaries outside of the reflex vertices are moved such that all $r$ points are now forming boundaries. 

\begin{figure}[h!]
    \centering
    \includegraphics[width=0.7\textwidth]{literature/concave_test.png}
    \caption{Generated simple polygon $P$ for $n = 12, r = 3$ \cite{maleki2022implementation} through Procedure 2.}
    \label{fig:arbitrary}
\end{figure}

The new algorithm is tested on simple polygons constructed as mentioned by starting from low$~r$ and gradually increasing it. The results are reported positively in the sense that the $|S|$ always remains close to the optimal, as a 2-approximation solution.

Therefore, through the newly implemented algorithm, Maleki and Mohades \cite{maleki2022implementation} testify that the algorithm of Ghosh \cite{GHOSH2010718} performs like a constant approximation in practice, and often better than its theoretical bound when tested on complex simple polygons.
% - guards placed on vertices = vertex guards
% - no restrictions = point guards
% - $P$ is called weak visibility polygon if every point in $P$ is visible from some point of an edge
% - the problem of computing a min number of guards is NP-hard
% - $O(n^4)$ approximation algorithm for computing $S$ (1)
% - $O(n^2)$ 6-approximation algorithm for vertex \subsection{Implementation of Polygon Guarding Algorithms for Art Gallery Problems \cite{maleki2022implementation}}
% - guards placed on vertices = vertex guards
% - no restrictions = point guards
% - $P$ is called weak visibility polygon if every point in $P$ is visible from some point of an edge
% - the problem of computing a min number of guards is NP-hard
% - $O(n^4)$ approximation algorithm for computing $S$ (1)
% - $O(n^2)$ 6-approximation algorithm for vertex guarding a weak vibisility $P$ with no holes (2)
% - implementation of algorithms and testing on weak visibility polygons
% - generating arbitrary weak visibility polygons *add algorithm* and testing them with $n = \overline{10, 15}$ and $r \in \{2, 3\}$ reflex vertices
	% - for low value of $n$ and $r$, it is better to use Algorithm 1 for minimising the number of vertex guards; since algorithm 2 is a constant approximation algorithm, algorithm 1 performs like a constant time approximation algorithm for small values of $n$ and $r$ experimentally
	% - since the criteria of minimisation is the number of guards rather than the running time which is a one-time affair unlike online algorithms, algorithm 1 is preferable even for weak visibility polygons
% - generating arbitrary weak visibility polygons and testing them with $n = \overline{10, 15}$ and $r$ reflex vertices close to the number of convex vertices
	% - for a low value of $n$ and $\frac n 2 \leq r$, algorithm 1 is better for guarding a weak visibility polygon with a min number of guards, because when the number of reflex vertices increases, the number of the diameter of the polygon and convex components decrease
% - generating arbitrary weak visibility polygons and testing them with large $n$
	% - algorithm 1 is better for guarding a weak visibility polygon with min number of guards, because it uses less guards than algorithm 2
% - generating arbitrary simply polygons *add algorithm* (Ghosh)
% - if $r << n$ , then the size of the optimal guard set is $\approx r$  => the number of edges $E$ in the visibility graph of such simple polygons is $O(n^2)$ => choose a small number $\log \log n$ as an upper bound for $r$ so that $r$ and optimal are close
% - if $r - c \leq \log \log n$, $c$ - small constant, then we place guards at all reflex vertices for guarding a simple polygon $P$, otherwise we place guards using the method of algorithm 1
% - even if $E$ reduces, the chosen guard sets remains close to the optimal and the algorithm assigns no more than twice the optimal number of guards
% => Ghosh's idea works better in practice and performs like a constant approximation algorithmalgorithm 2 is a constant approximation algorithm, algorithm 1 performs like a constant time approximation algorithm for small values of $n$ and $r$ experimentally
	% - since the criteria of minimisation is the number of guards rather than the running time which is a one-time affair unlike online algorithms, algorithm 1 is preferable even for weak visibility polygons
% - generating arbitrary weak visibility polygons and testing them with $n = \overline{10, 15}$ and $r$ reflex vertices close to the number of convex vertices
	% - for a low value of $n$ and $\frac n 2 \leq r$, algorithm 1 is better for guarding a weak visibility polygon with a min number of guards, because when the number of reflex vertices increases, the number of the diameter of the polygon and convex components decrease
% - generating arbitrary weak visibility polygons and testing them with large $n$
	% - algorithm 1 is better for guarding a weak visibility polygon with min number of guards, because it uses less guards than algorithm 2
% - generating arbitrary simply polygons *add algorithm* (Ghosh)
% - if $r << n$ , then the size of the optimal guard set is $\approx r$  => the number of edges $E$ in the visibility graph of such simple polygons is $O(n^2)$ => choose a small number $\log \log n$ as an upper bound for $r$ so that $r$ and optimal are close
% - if $r - c \leq \log \log n$, $c$ - small constant, then we place guards at all reflex vertices for guarding a simple polygon $P$, otherwise we place guards using the method of algorithm 1
% - even if $E$ reduces, the chosen guard sets remains close to the optimal and the algorithm assigns no more than twice the optimal number of guards
% => Ghosh's idea works better in practice and performs like a constant approximation algorithm
\input{2d_summaries}