\section{Introduction}

\subsubsection*{Reinforcement Learning}
Reinforcement learning (RL) has long been applied to board games, like chess, backgammon, and more recently Go. The paper "Playing Atari with Deep Reinforcement Learning" \cite{Atari_DQN_2013} from 2013 revolutionized RL in video games, where an agent surpasses humans on the Atari console from 1977 on multiple games by performing actions based on screen observations. Currently, researchers at OpenAI and Deepmind successfully train their intelligent agents on complex games like Dota 2 and StarCraft. In addition to games, reinforcement learning is also promising in areas such as robotics, stock market trading, and recommendation systems \cite{web:RL_Applications}.
\\[2.5mm]
Machine learning deals with computer algorithms that improve themselves over time from experience. Reinforcement learning is a type of machine learning where an intelligent agent interacts with an environment. The agent receives feedback from the environment and improves itself by trial and error. This feedback is often called the reward and can be represented by a positive or negative number. The agent adjusts its actions to obtain the highest reward from all upcoming states. This total remaining reward is the \textit{return} which the agent learns to optimize. The book \textit{Reinforcement Learning: An Introduction} by Sutton and Barto \cite{RL_Book_Sutton_Barto} is heavily consulted in this thesis and contains the basic ideas of reinforcement learning.


\subsubsection*{Slither Game}

In this thesis, machine learning agents learn to play the competitive online browser game Slither.io. Slither has similarities with the classic Snake game. In both Snake and Slither the player controls a snake. The goal in these games is to grow in size by eating food pellets located around the arena. In Slither, there are hundreds of competing snakes in a single arena. When the head of a snake collides with the tail of another snake it is game-over for the headbutting snake. This snake drops pellets where it dies. This makes it dangerous to stay close to other snakes, but also rewarding when the other snakes die.
\\[2.5mm]
The main goal of this thesis is to create a superhuman agent in Slither. The Slither agent is first trained inside a replicated version of Slither in the game-engine \textit{Unity}. Unity has a machine learning package called \textit{ML-Agents} which contains an API for communication with \textit{Python}. The algorithms are implemented with the help of Python's machine learning library \textit{PyTorch} and computer vision library \textit{OpenCV}. Eventually, the agent will be deployed onto the browser game to test itself against real players. These performances are compared for different algorithms and techniques.
\\[2.5mm]
The initial research and work is presented as preparation for the thesis project. In the next chapter 2: 'Slither.io Game', the features of the game are explored. In  chapter 3: 'Research and Literature' the relevant theory is explained in detail. The main ideas of recent useful papers are presented. In the final chapter 4: 'Project Details'  the thesis plans and goals are described, along with a roadmap for the next months. Similar projects are explored and the current progress on the Slither implementation into Unity is shown. 

